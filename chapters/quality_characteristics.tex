\chapter{软件质量特性}
<Specify any additional quality characteristics for the project that will be important to either the customers or the developers. Some to consider are: adaptability, availability, correctness, flexibility, interoperability, maintainability, portability, reliability, reusability, robustness, testability, and usability. Write these to be specific, quantitative, and verifiable when possible. At the least, clarify the relative preferences for various attributes, such as ease of use over ease of learning.

详细说明项目任何其他的质量特性。该特性对客户和开发者都非常重要。考虑的方面包括:适应性,可用性,正确性,灵活性,交互工作能力,可维护性,可移植性,可靠性,可重用性,鲁棒性,可测试性和可用性等。定量的详细描述这些特性,尽可能的可验证。对不同属性之间的重要性加以阐述,如:易用性比易学性更重要。

<Please use the below sub-section for each attributes separately. You can copy the section for additional attributes. >

每一个属性单独使用一个小节描述,可根据需要进行增减,如增加可维护性小节等。

\section {适应性}
\proname 的网页应可以在桌面和移动平台上访问,适配几乎所有平台的所有浏览器。\proname 的应用应可以在各种版本系统的电脑和手机上运行。\proname 的应用应支持 Android 系统的车载娱乐系统和机顶盒。

\section {可用性}
\proname 的网页客户端应具有简洁明了的UI,电脑和手机客户端应提供直观的 UI 和尽可能简单的菜单,避免菜单的多层嵌套和过多的菜单按键,使得其使用便利快捷。

\section {易学性}
\proname 的网页客户端应简洁明了,使得任何具有基本浏览网页技能的用户都可以轻松学习使用。 \proname 的客户端可以附有对首次使用的用户的指引。

\section {正确性}
\proname 的服务器端应对客户端的请求使用数据包校验和用户信息的比对,拒绝非法的请求。
\proname 的客户端在做出对用户账户的获取和变动之前,应与服务器通信比对用户凭据是否有效。

\section {灵活性}
\proname 的客户端应减少各个模块间的依赖,保证逻辑的简洁,并提供简单、全面的接口。使得开发和维护过程中可以根据需求的变更快速更新。

\section {可维护性}
\proname 的代码应有良好的文档和丰富的注释(占总代码的至少 30\%),加上简洁的逻辑,使得整个服务易于维护。

\section {可移植性}
\proname 的网页客户端代码应使用标准的 HTML5 和 Javascript ,而尽量避免使用浏览器特有的 API ;使用响应性网页设计;从而使得在任意分辨率和任意支持 HTML5 的浏览器上都能正常显示。 \proname 的本地客户端应尽可能少的使用最新的和过时的 API ;可以使用跨平台的框架(如 Qt ),并在操作系统变动时能尽可能地减小修改。

\section {可测试性}
\proname 的各个模块间应尽可能地减少外部和内部的依赖。如有依赖,应保证尽可能地使用稳定的 API ,并可以使用 mock object 实现单元测试。

\section {健壮性}
\proname 的客户端和服务器端应检测可能发生的错误并处理。服务器端应保证前端和数据库的隔离,并防止 SQL 注入等攻击方式。

\section {可靠性}
\proname 服务器的数据库应支持高并发的访问,客户端应支持在离线时的正常使用(播放已下载的歌曲),而服务器应有多台,并保证失败时的 failover 。

