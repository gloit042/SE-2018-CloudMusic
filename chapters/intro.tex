\chapter{简介}
\section{目的}

本 需求分析文档的阅读对象是\proname 的开发者、开放接口使用者和普通用户。
本说明书通过图文结合的介绍性文字,向开发者,特别是网页前端开发人员,介绍
    了在\proname 的开发过程中需要完成的功能要点,为开发人员实现\proname
    的网页端以及客户端提供了帮助。

对于\proname 的开放应用程序接口(API)的使用者,通过本文档的功能说明部分,
    尤其是\emph{外部接口需求}一节,
    可以详细地了解我们提供的开放接口,并通过我们提供的实例以及规范,
    来学习如何通过这些开放接口获取歌曲的各种元信息,
    并获得在我们数据库中的用户评分、歌曲排名、受欢迎程度、热销程度
    等实用的信息,并将它们用在自己的网页或应用中。

对于\proname 的用户,需求分析文档同样可以向他们提供尽可能的使用上的
    帮助,通过研读\emph{总体概述}章节,可以以一个上层建筑的角度
    总览\proname 项目,其中,\emph{软件功能}小节可以帮助用户了解
    本项目的所有功能,而不需要关心实现细节、数据接口、性能要求
    等与客户使用场景无关的细节信息。

当然,对于计算机学科的学习者,或者是潜在的产品研发部门的学习者,
    同样可以通过本篇说明书学习与软件设计、项目规划相关的知识。

\section{范围}

本篇产品需求分析文档包含以下内容:

\begin{enumerate}
    \item \proname 项目(下称本项目)的总体项目概述
    \item 本项目的软件功能概述
    \item 本项目的目标用户以及用户背景分析
    \item 本项目具体的设计要求,包括:
    \begin{enumerate}
        \item 功能上的需求
        \item 软件性能上的需求
        \item 外部应用程序接口(API)的设计
    \end{enumerate}
    \item 本项目总体设计约束和软件质量特性
    \item 本项目与其他项目以及库的依赖关系
    \item 开发本项目的具体计划
    \item 在开发本项目的过程中的注意事项
    \item 在本项目开发过程中,各功能点的优先级
\end{enumerate}