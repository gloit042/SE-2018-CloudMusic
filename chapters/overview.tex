\chapter{总体概述}

\section{软件概述}
\subsection{项目介绍}

音乐使人们生活中不可或缺的一部分,
    而市面上,已经有许多成熟的音乐播放系统,
    为了达到产品生存与盈利,我们需要不同点与亮点。
我们的\proname 项目有两大竞争优势。

首先,本项目借助云服务器,达到用户“一次购买,随时随地,想听就听”,
    只要用户购买了歌曲,或者订阅了会员服务,
    就可以在任何时间、任何地点方便地收听到他有权限访问的歌曲。
为了达到这一点,我们需要强大的云服务器作为后端,
    并使得用户的购买记录能够安全、永久地被存储,
    并且,为了让用户能够尽可能在任何情况下收听到他购买的歌曲,
    我们需要为各种浏览器环境优化网页环境,
    并且在各种个人电脑系统上推出客户端,
    同时,各品牌、各平台、各系统的移动客户端也要跟上
    当今移动设备快速的迭代速度。

其次,我们的个性推荐服务能够让用户体验到“\proname 比你更懂你想听什么”
    的感受。
    我们需要建立用户大数据模型,根据用户的试听、购买、订阅行为,
    为用户加上各种标签,通过适配标签集与歌曲集之间的相关性,
    精确地将用户喜爱的歌曲推荐给用户。
这既可以增加用户粘度,又可以使用户受到推荐之后,购买更多的歌曲。

\subsection{产品环境介绍}

\begin{itemize}
    \item 我们的项目分为服务端与客户端,其中,服务端是统一的,
    所有客户端都由该统一服务端进行数据交互与信息支持;
    而为了满足用户多种多样的收听音乐的环境,
    我们的客户端有较多种类,
    目前,我们需要考虑以下客户端环境:
    \begin{enumerate}
        \item 网页浏览器环境,对应网页客户端;
        \item Windows环境,对应传统程序客户端和较新的
        Universal Application(应用商店内应用程序);
        \item Mac OS环境,对应一般程序客户端;
        \item Linux环境,对应一般客户端以及基于命令行的最小化客户端;
        \item iOS移动环境,对应iOS应用(App Store内应用程序);
        \item Android移动环境,对应安卓应用程序;
        \item 车载娱乐系统环境,对应厂商预装车载娱乐系统内程序;
        \item 电视机机顶盒环境,对应电视服务商提供的机顶盒内程序;
    \end{enumerate}
    \item
    \item
\end{itemize}

% 描述的是本产品与其它产品或项目所组成的整体环境。

% 1.如果本产品是独立的并完全自我包含,在此说明这一点。

% 2.如果SRS定义的产品是更大的系统或项目的组件(此种情形经常发生),那么应:

% 	A. 描述此大系统或项目每个组件的功能,并且标识接口。

% 	B.  确定本软件产品主要外部接口。( 注意:在此部分并不进行这些接口的详细描述;对这些接口的详细描述在SRS的其它 部分提供。)

%     C. 描述相关产品硬件和所使用的外部设备。(  注意:  这只是概述性描述。)

% 通过方块图来描述大系统或项目的主要组件,互连性以及外部接口将是非常有帮助的。本部分不应提出一个具体的设计解决方案或对解决方案的具体设计约束(具体设计约束将在具体需求章节中描述)。本部分内容是产生设计约束的基础。

\section{软件功能}
Summarizes the major functions that must be implemented through the software, and the functions to be implemented through user operation. Details will be provided in the Specific Requirement, so only a summary (such as a directory list) is needed here. The functions should be organized to make them understandable to the readers, and be appropriate for subsequent design and tests. Diagrams like top-level data flow diagram or object class diagram are recommended to illustrate the relationships among the major requirement groups

Sometimes, this section can directly refer to the superior specification of the software that allocate the specific requirements to this software ( if existed ).
The specific requirements should not be described in this section. But this section is the basis of the specific requirements.

概述软件的必须实现的和通过用户操作实现的主要功能。这里只需要进行简要描述(例如目录列表),详细描述在详细需求部分描述。对需求功能进行组织,以便于读者理解,并能指导后续的设计和测试。可以用图表来表示主要需求群组之间的关系,例如:高层的数据流图,面向对象的分析等。

有时此部分所要求的功能概述可以从分配具体功能给此软件产品的更高层规格(如果存在的话)直接引用。

本节不应描述具体需求。但本节内容是具体需求章节的基础。

\section{用户特征}
List down the basic required characteristics of the user or operator of the system. E.g. the experience, Skill level, required role etc.,
This part should not describe the specific requirements, instead, it provides the basis for the specific requirements.

列出对用户或系统操作者的要求,如:经验,能力,角色等。

本节不应描述具体需求。但本节内容是具体需求章节的基础。

\section{假设和依赖关系}
List any assumed factors (as opposed to known facts) that could affect the requirements stated in the SRS. These could include third party or commercial components that you plan to use, issues around the development or operating environment, or constraints. The project could be affected if these assumptions are incorrect, are not shared, or change. Also identify any dependencies the project has on external factors, such as software components that you intend to reuse from another project, unless they are already documented elsewhere (for example, in the vision and scope document or the project plan).

列出可能影响SRS中需求的所有的假设因素(与已知事实相对而言),包括准备使用的第三方或商业组件,操作和开发环境的问题约束等。如果上述假设不正确、没有被告知或者改变了都将对项目产生影响。列出项目对外部条件的依赖,例如重用其他项目的模块等。如果在其他文档(例如项目计划或范围文档等)里已经描述了,在这里可以不用描述。
