\chapter{具体需求}


\section{功能需求}

本子章节详细列出了我们产品的各个功能点的输入怎样被转换成输出,
	并描述了软件必须执行的基本动作。

\subsection{R.CLOUDMUSIC.SYS.001客户端启动}
\subsubsection{介绍}
	这是在客户端启动时,我们需要处理的功能点,包括:
	\begin{itemize}
		\item 检查更新,若有更新,应当通知用户;
		\item 进行主页初始化(R.CLOUDMUSIC.APP.001),显示主页(Home)界面的信息,供用户使用;
		\item 进行账户自动登录(R.CLOUDMUSIC.USER.001),如果上次关闭时已登录,
			则使用上一次的登录信息尝试自动登录;
		\item 进行音乐集列表同步(TODO),如果登录成功,需要对用户主页显示的音乐集做
			云同步;
		\item 对于部分客户端类型,展示APP开屏,用于等待初始化时的信息展示,
			也可展示广告;
	\end{itemize}
\subsubsection{输入}
	\begin{itemize}
		\item \textbf{输入来源}:客户端所在系统或浏览器;
		\item \textbf{内容}:用户打开了对应程序或网页,系统或浏览器发出的响应需求请求;
		\item \textbf{有效输入范围}:需判定为合法的请求;
	\end{itemize}
\subsubsection{处理}
	\begin{enumerate}
		\item 检测输入有效性,对于软件客户端,检测请求来源和格式是否符合标准,
			对于网页客户端,检测请求是否合法,否则请传输错误网页或特定网页
			(如404网页);
		\item 检查更新,若有更新,应当通知用户,若用户同意升级,则进行升级
			(详见R.CLOUDMUSIC.SYS.003);
		\item 进行主页初始化,详细请见 R.CLOUDMUSIC.APP.001;
		\item 若上次关闭时,用户登录了账户且网络已连接,则使用上次登录时的账户信息与
			缓存的登陆信息,尝试进行自动登陆(登录功能请见R.CLOUDMUSIC.USER.001)
			若成功,则进行音乐集的云同步(详细功能需求请见XXXX)
			若不成功,需要告知用户,进行重新手动登陆,并展示缓存在本地的主页配置;
		\item 如果是移动客户端,在以上过程中,需要使用APP开屏来在加载期间向用户展示
			歌曲推荐内容或广告,展示的内容是事先保存的,而在展示的过程中,也要
			联网向服务器查询同步下一次要向用户展示的APP开屏图片;
	\end{enumerate}
	\noindent 异常处理:
	\begin{enumerate}
		\item \textbf{网络错误}:不进行自动登陆操作、不进行APP开屏同步数据、不进行音乐集同步;
		\item \textbf{登录失败}:通知用户重新登录、不进行音乐集同步;
		\item \textbf{初始化失败}:返回错误信息;
	\end{enumerate}
\subsubsection{输出}
\begin{itemize}
	\item \textbf{输出位置}:响应系统的请求或响应浏览器的请求;
	\item \textbf{内容}:通过具体响应,告知启动初始化完成;
	\item \textbf{有效输出范围}:按照系统响应请求的方式或者网络服务发出响应报文的
		正确格式决定相关输出;
	\item \textbf{错误消息}:返回错误信息,具体方式按照系统响应请求的方式
		或者网络服务发出响应报文的格式;
\end{itemize}

\subsection{R.CLOUDMUSIC.SYS.002客户端关闭}
\subsubsection{介绍}
	这是在客户端关闭时,我们需要处理的功能点,包括:
	\begin{itemize}
		\item 检查是否有下载任务,若有,询问用户;
		\item 暂停所有的下载服务以及流服务;
		\item 最后进行一次同步数据(TODO);
		\item 关闭程序
	\end{itemize}
\subsubsection{输入}
	\begin{itemize}
		\item \textbf{输入来源}:客户端所在系统或浏览器;
		\item \textbf{内容}:用户或系统关闭了对应程序或网页,系统或浏览器发出的响应需求请求;
		\item \textbf{有效输入范围}:需判定为合法的请求;
	\end{itemize}
\subsubsection{处理}
	\begin{enumerate}
		\item 检测输入有效性,对于软件客户端,检测请求来源和格式是否符合标准,
			对于网页客户端,检测请求是否是合法的断开连接请求;
		\item 检测是否有正在运行的下载服务,如果有,询问用户是否任要关闭;
		\item 断开下载服务(如果有);
		\item 断开流数据服务(如果有);
	\end{enumerate}
	\noindent 异常处理:
	\begin{enumerate}
		\item \textbf{网络错误}:告知用户同步失败,询问是否重试,
			若用户选择重试,则重试,否则,放弃同步;
	\end{enumerate}
\subsubsection{输出}
\begin{itemize}
	\item \textbf{输出位置}:响应系统的请求或响应浏览器的请求;
	\item \textbf{内容}:通过具体响应,告知关闭完成;
	\item \textbf{有效输出范围}:按照系统响应请求的方式或者网络服务发出响应报文的
		正确格式决定相关输出;
	\item \textbf{错误消息}:返回错误信息,具体方式按照系统响应请求的方式
		或者网络服务发出响应报文的格式;
\end{itemize}

\subsection{R.CLOUDMUSIC.SYS.003客户端升级}
\subsubsection{介绍}
	这是在客户端升级时(网页客户端不考虑),我们需要处理的功能点,包括:
	\begin{itemize}
		\item 向服务端请求更新所需的文件并校验升级文件;
		\item 应用升级并重启;
	\end{itemize}
\subsubsection{输入}
	\begin{itemize}
		\item \textbf{输入来源}:客户端;
		\item \textbf{内容}:用户主动要求升级或者同意了自动检测的升级请求;
		\item \textbf{有效输入范围}:需判定为合法的请求;
	\end{itemize}
\subsubsection{处理}
	\begin{enumerate}
		\item 检测输入有效性,检测请求来源和格式是否符合标准;
		\item 检测是否有正在运行的下载服务,如果有,询问用户是否任然要升级;
		\item 进行标准的关闭程序操作(详见R.CLOUDMUSIC.SYS.002);
		\item 启动升级程序,进行升级包下载,并向多个服务器请求校验信息;
		\item 进行文件校验;
		\item 按照程序预先设定的文件更新方式进行升级;
		\item 完成后,重启程序(详见R.CLOUDMUSIC.SYS.001);
	\end{enumerate}
	\noindent 异常处理:
	\begin{enumerate}
		\item \textbf{网络错误}:告知用户升级失败,询问是否重试,
			若用户选择重试,则重试,否则,放弃升级,直接重启原来的版本;
		\item \textbf{校验错误}:同\textbf{网络错误}的处理方式;
	\end{enumerate}
\subsubsection{输出}
\begin{itemize}
	\item \textbf{输出位置}:客户端;
	\item \textbf{内容}:通过具体响应,告知升级完成;
	\item \textbf{有效输出范围}:无限制(根据客户端的程序逻辑,不会产生非法响应);
	\item \textbf{错误消息}:返回错误信息,告知客户端升级失败的原因,并通过客户端告知用户;
\end{itemize}

\subsection{R.CLOUDMUSIC.USER.001用户登陆}
\subsubsection{介绍}
	这是在 收到用户登录请求 时,我们需要处理的功能点,包括:
	\begin{itemize}
		\item 检测登录的凭证是否合法;
		\item 返回结果并更新凭证信息;
	\end{itemize}
\subsubsection{输入}
	\begin{itemize}
		\item \textbf{输入来源}:客户端或网页浏览器;
		\item \textbf{内容}:请求来源、账户名称、登陆凭据(详见下述);
		\item \textbf{有效输入范围}:需判定为合法的登陆请求;
	\end{itemize}
	\noindent 其中,登陆凭据的具体内容可以为加密后的密码信息,或者是上一次登录后
		服务器传回的登陆凭据;
\subsubsection{处理}
	\begin{enumerate}
		\item 检测输入有效性,检测请求来源和格式是否符合标准;
		\item 检查登陆凭据的合法性,若为加密后的密码,解密后进行校验,若成功,进入下一步,
			否则产生校验失败的异常;若为上一次登录后服务器传回的登陆凭据,
			则检测其是否已超时或失效,若有效且合法,进入下一步,否则产生登录超时的异常。
		\item 生成登陆凭据,在服务器上记录该凭据;
		\item 返回凭据;
	\end{enumerate}
	\noindent 异常处理:
	\begin{enumerate}
		\item \textbf{校验失败}:返回校验信息失败错误的信息,客户端将请求用户重试
			或尝试找回密码;
		\item \textbf{登录超时}:返回登录超时的信息,客户端将请求用户手动登陆;
	\end{enumerate}
\subsubsection{输出}
\begin{itemize}
	\item \textbf{输出位置}:客户端或网页浏览器;
	\item \textbf{内容}:若成功,返回登录凭据(对于网页客户端,是Cookie),
		若失败,返回登陆失败的原因;
	\item \textbf{有效输出范围}:无限制(根据服务端的程序逻辑,不会产生非法响应);
	\item \textbf{错误消息}:返回错误信息,告知用户登陆失败的原因,并按照情况处理;
\end{itemize}

\subsection{R.CLOUDMUSIC.USER.002用户注册}
\subsubsection{介绍}
	这是在 收到用户注册请求 时,我们需要处理的功能点,包括:
	\begin{itemize}
		\item 检测注册的相关信息是否合法;
		\item 在用户数据库中添加用户相关信息;
		\item 返回结果;
	\end{itemize}
\subsubsection{输入}
	\begin{itemize}
		\item \textbf{输入来源}:客户端或网页浏览器;
		\item \textbf{内容}:请求来源、注册信息(详见下述);
		\item \textbf{有效输入范围}:需判定为合法的注册请求;
	\end{itemize}
	\noindent 其中,注册信息的具体内容包括:
	\begin{itemize}
		\item 用户名(Username) 
		\item 密码(Password) 
		\item 邮箱(Email) 
	\end{itemize}
\subsubsection{处理}
	\begin{enumerate}
		\item 检测输入有效性,检测请求来源和格式是否符合标准;
		\item 检查使用的用户名和邮箱是否在数据库中有重复,使用的字符和长度是否
			是符合要求的,若有重复或不合法,则返回相应失败信息;
		\item 检查使用的密码,是否过于简单,若是,则返回失败信息;
		\item 返回注册结果;
	\end{enumerate}
	\noindent 异常处理:
	\begin{enumerate}
		\item \textbf{用户名或邮箱不合法}:
		返回不合法情况的相关失败错误的信息,客户端将请求用户重新按照要求注册;
		\item \textbf{密码过于简单}:返回错误的信息,客户端将请求用户使用更复杂的密码;
	\end{enumerate}
\subsubsection{输出}
\begin{itemize}
	\item \textbf{输出位置}:客户端或网页浏览器;
	\item \textbf{内容}:若成功,返回成功信息,若失败,返回注册失败的原因;
	\item \textbf{有效输出范围}:无限制(根据服务端的程序逻辑,不会产生非法响应);
	\item \textbf{错误消息}:返回错误信息,告知用户注册失败的原因,并按照情况处理;
\end{itemize}

\subsection{R.CLOUDMUSIC.USER.003用户注销}
\subsubsection{介绍}
	这是在 响应用户注销 时,我们需要处理的功能点,包括:
	\begin{itemize}
		\item 在服务端记录注销的信息,无效化相关登陆凭据;
	\end{itemize}
\subsubsection{输入}
	\begin{itemize}
		\item \textbf{输入来源}:客户端或网页浏览器;
		\item \textbf{内容}:请求来源,注销的用户信息;
		\item \textbf{有效输入范围}:需判定为合法的登陆请求;
	\end{itemize}
\subsubsection{处理}
	\begin{enumerate}
		\item 检测输入有效性,检测请求来源和格式是否符合标准;
		\item 检查注销请求是否合法,用户名是否存在;
		\item 在服务器上无效化相应的用户登陆凭据;
		\item 返回结果;
	\end{enumerate}
	\noindent 异常处理:
	\begin{enumerate}
		\item \textbf{校验失败}:返回注销失败的信息,客户端将请求重试,放弃注销;
	\end{enumerate}
\subsubsection{输出}
\begin{itemize}
	\item \textbf{输出位置}:客户端或网页浏览器;
	\item \textbf{内容}:若成功,返回注销成功的信息,
		若失败,返回注销失败的原因;
	\item \textbf{有效输出范围}:无限制(根据服务端的程序逻辑,不会产生非法响应);
	\item \textbf{错误消息}:返回错误信息,告知用户注销失败的原因,并按照情况处理;
\end{itemize}

\subsection{R.CLOUDMUSIC.APP.001生成主页界面}
\subsubsection{介绍}
	这是在 生成主页(Home)界面 时,我们需要处理的功能点,包括:
	\begin{itemize}
		\item 生成主页界面的内容;
	\end{itemize}
\subsubsection{输入}
	\begin{itemize}
		\item \textbf{输入来源}:客户端或网页浏览器;
		\item \textbf{内容}:客户端的状态或者网页的状态(如网页尺寸等);
		\item \textbf{有效输入范围}:需判定为合法的请求;
	\end{itemize}
\subsubsection{处理}
	\begin{enumerate}
		\item 检测输入有效性,检测请求来源和格式是否符合标准;
		\item 生成界面元素,具体的界面要求,请参照小节\ref{ssec:ui};
		\item 返回完成的信息;
	\end{enumerate}
\subsubsection{输出}
\begin{itemize}
	\item \textbf{输出位置}:客户端或网页浏览器;
	\item \textbf{内容}:告知生成完成;
	\item \textbf{有效输出范围}:无限制(根据服务端的程序逻辑,不会产生非法响应);
	\item \textbf{错误消息}:正常情况下不会产生非法响应,若有,则应视为程序BUG;
\end{itemize}

\subsection{R.CLOUDMUSIC.APP.002生成音乐集界面}
\subsubsection{介绍}
	这是在 生成音乐集界面 时,我们需要处理的功能点,包括:
	\begin{itemize}
		\item 判定音乐集是否可以被查看;
		\item 生成音乐集界面的内容;
	\end{itemize}
\subsubsection{输入}
	\begin{itemize}
		\item \textbf{输入来源}:客户端或网页浏览器;
		\item \textbf{内容}:请求的音乐集ID以及用户登录凭证(可选);
		\item \textbf{有效输入范围}:需判定为合法的请求;
	\end{itemize}
\subsubsection{处理}
	\begin{enumerate}
		\item 检测输入有效性,检测请求来源和格式是否符合标准;
		\item 检测登陆凭据,若校验失败或登录超时,返回对应错误,不进入下一步;
		\item 检查音乐集ID是否存在,若不存在,返回对应错误,不进入下一步;
		\item 检查音乐集的可访问性,若为私密(private)音乐集,
			检查用户凭证对应的账户是否对其有查看权,若否,返回对应错误,不进入下一步;
		\item 生成音乐集界面元素,具体的界面要求,请参照小节\ref{ssec:ui};
		\item 返回完成的信息;
	\end{enumerate}
	\noindent 异常处理:
	\begin{enumerate}
		\item \textbf{登录信息失败}:返回登陆失败错误的信息,客户端将请求用户重登陆;
		\item \textbf{音乐集ID不存在}:返回音乐集不存在的信息,客户端将展示错误页面;
		\item \textbf{音乐集不可访问}:返回音乐集不可访问性的信息,
			客户端将展示无法访问私密音乐集页面,若没有登录,询问是否登陆;
	\end{enumerate}
\subsubsection{输出}
\begin{itemize}
	\item \textbf{输出位置}:客户端或网页浏览器;
	\item \textbf{内容}:若成功,返回 音乐集界面生成完成的信息 ,若失败,返回 失败的原因;
	\item \textbf{有效输出范围}:无限制(根据服务端的程序逻辑,不会产生非法响应);
	\item \textbf{错误消息}:返回错误信息,告知用户查看失败的原因,并按照情况处理;
\end{itemize}

\subsection{R.CLOUDMUSIC.APP.003生成音乐播放界面}
\subsubsection{介绍}
	这是在 生成音乐播放界面 时,我们需要处理的功能点,包括:
	\begin{itemize}
		\item 判定音乐是否可以被查看;
		\item 生成音乐播放界面的内容;
	\end{itemize}
\subsubsection{输入}
	\begin{itemize}
		\item \textbf{输入来源}:客户端或网页浏览器;
		\item \textbf{内容}:请求的音乐ID以及用户登录凭证(可选);
		\item \textbf{有效输入范围}:需判定为合法的请求;
	\end{itemize}
\subsubsection{处理}
	\begin{enumerate}
		\item 检测输入有效性,检测请求来源和格式是否符合标准;
		\item 检测登陆凭据,若校验失败或登录超时,返回对应错误,不进入下一步;
		\item 检查音乐ID是否存在,若不存在,返回对应错误,不进入下一步;
		\item 检查音乐的可访问性,若为付费的音乐,
			检查用户凭证对应的账户是否对其有播放权,
			若否或没有登录,检查该音乐是否允许试听,若可以,生成试听模式的界面,
			否则返回错误信息;
		\item 若具有播放权,生成完整播放界面;
		\item 具体的两种界面的要求,请参照小节\ref{ssec:ui};
		\item 返回完成的信息;
	\end{enumerate}
	\noindent 异常处理:
	\begin{enumerate}
		\item \textbf{登录信息失败}:返回登陆失败错误的信息,客户端将请求用户重登陆;
		\item \textbf{音乐ID不存在}:返回音乐不存在的信息,客户端将展示错误页面;
		\item \textbf{音乐不可访问}:返回音乐集不可访问性的信息,客户端提示是否要购买;
	\end{enumerate}
\subsubsection{输出}
\begin{itemize}
	\item \textbf{输出位置}:客户端或网页浏览器;
	\item \textbf{内容}:若成功,返回 音乐播放界面生成完成的信息 ,若失败,返回 失败的原因;
	\item \textbf{有效输出范围}:无限制(根据服务端的程序逻辑,不会产生非法响应);
	\item \textbf{错误消息}:返回错误信息,告知用户尝试进入播放界面失败的原因,
		并按照情况处理;
\end{itemize}

\subsection{R.CLOUDMUSIC.APP.004生成音乐推荐界面}
\subsubsection{介绍}
	这是在 生成音乐推荐界面 时,我们需要处理的功能点,包括:
	\begin{itemize}
		\item 判定用户的登录状态;
		\item 生成音乐推荐界面的内容;
	\end{itemize}
\subsubsection{输入}
	\begin{itemize}
		\item \textbf{输入来源}:客户端或网页浏览器;
		\item \textbf{内容}:请求的来源以及用户登录凭证(可选);
		\item \textbf{有效输入范围}:需判定为合法的请求;
	\end{itemize}
\subsubsection{处理}
	\begin{enumerate}
		\item 检测输入有效性,检测请求来源和格式是否符合标准;
		\item 检测登陆凭据,若校验失败或登录超时,返回对应错误,不进入下一步;
		\item 向服务器请求个性化推荐以及音乐排名等信息,如果没有登录,
			则仅仅生成音乐排名的信息,不生成个性化内容;
		\item 生成音乐推荐界面的内容,具体的界面要求,请参照小节\ref{ssec:ui};
		\item 返回完成的信息;
	\end{enumerate}
	\noindent 异常处理:
	\begin{enumerate}
		\item \textbf{登录信息失败}:返回登陆失败错误的信息,客户端将请求用户重登陆;
		\item \textbf{网络访问}:返回网络错误的信息,客户端提示无法查看,
			请求用户检查网络状态;
	\end{enumerate}
\subsubsection{输出}
\begin{itemize}
	\item \textbf{输出位置}:客户端或网页浏览器;
	\item \textbf{内容}:若成功,返回 音乐推荐界面生成完成的信息 ,若失败,返回 失败的原因;
	\item \textbf{有效输出范围}:无限制(根据服务端的程序逻辑,不会产生非法响应);
	\item \textbf{错误消息}:返回错误信息,告知用户尝试进入推荐界面失败的原因,
		并按照情况处理;
\end{itemize}

\subsection{R.CLOUDMUSIC.APP.005生成账户界面}
\subsubsection{介绍}
	这是在 生成账户界面 时,我们需要处理的功能点,包括:
	\begin{itemize}
		\item 判定用户的登录状态;
		\item 生成对应的账户界面的内容;
	\end{itemize}
\subsubsection{输入}
	\begin{itemize}
		\item \textbf{输入来源}:客户端或网页浏览器;
		\item \textbf{内容}:请求的来源以及用户登录凭证(可选);
		\item \textbf{有效输入范围}:需判定为合法的请求;
	\end{itemize}
\subsubsection{处理}
	\begin{enumerate}
		\item 检测输入有效性,检测请求来源和格式是否符合标准;
		\item 若没有登陆,生成登陆界面,也要提供注册功能的入口;
		\item 检测登陆凭据,若校验失败或登录超时,返回对应错误,不进入下一步;
		\item 向服务器请求用户的数据;
		\item 生成用户的界面信息元素,具体的界面要求,请参照小节\ref{ssec:ui};
		\item 返回完成的信息;
	\end{enumerate}
	\noindent 异常处理:
	\begin{enumerate}
		\item \textbf{登录信息失败}:返回登陆失败错误的信息,客户端将请求用户重登陆;
		\item \textbf{网络访问}:返回网络错误的信息,客户端提示无法查看,
			请求用户检查网络状态;
	\end{enumerate}
\subsubsection{输出}
\begin{itemize}
	\item \textbf{输出位置}:客户端或网页浏览器;
	\item \textbf{内容}:若成功,返回 账户界面生成完成的信息 ,若失败,返回 失败的原因;
	\item \textbf{有效输出范围}:无限制(根据服务端的程序逻辑,不会产生非法响应);
	\item \textbf{错误消息}:返回错误信息,告知用户尝试进入账户界面失败的原因,
		并按照情况处理;
\end{itemize}

\subsection{R.CLOUDMUSIC.APP.005生成设置界面}
\subsubsection{介绍}
	这是在 生成设置界面 时,我们需要处理的功能点,包括:
	\begin{itemize}
		\item 生成设置界面的内容;
	\end{itemize}
\subsubsection{输入}
	\begin{itemize}
		\item \textbf{输入来源}:客户端或网页浏览器;
		\item \textbf{内容}:请求的来源;
		\item \textbf{有效输入范围}:需判定为合法的请求;
	\end{itemize}
\subsubsection{处理}
	\begin{enumerate}
		\item 检测输入有效性,检测请求来源和格式是否符合标准;
		\item 读取当前的用户设置信息;
		\item 生成对应的设置界面,请参照小节\ref{ssec:ui};
		\item 返回完成的信息;
	\end{enumerate}
\subsubsection{输出}
\begin{itemize}
	\item \textbf{输出位置}:客户端或网页浏览器;
	\item \textbf{内容}:若成功,返回 设置界面生成完成的信息 ,若失败,返回 失败的原因;
	\item \textbf{有效输出范围}:无限制(根据服务端的程序逻辑,不会产生非法响应);
	\item \textbf{错误消息}:理论上,该界面不应该发生错误,若有,则应当视为程序BUG;
\end{itemize}



\section{性能需求}
<If there are performance requirements, state them here and explain their rationale, to help the developers understand the intent and make suitable design choices. Specifies the timing relationships for real time systems. Such requirements should be made as specific as possible. >

如果有性能方面的需求,在这里列出并解释他们的原理。以帮助开发者理解意图以做出正确的设计选择。在实时系统中的时序关系。保证需求尽可能的详细而精确。


\subsection{性能需求1}
Describes the statically and dynamically quantized requirements on the software (or the interaction between user and the software)
Static quantized requirement could include:
A. Maximum number of terminal supported.
B. Maximum number of users that can use the software at the same time.
C. Maximum number of files and records to be processed
D. Maximum size of  tables and files
Dynamically quantized requirements could include:
A. Specific duration of normal value and peak value of workload (e.g., one hour)
B. Number of event and task and data volume to be processed 
All these requirements should be described by measurable term, for example, saying "95% of the events should be processed in 1 second", instead of saying "the operator need not wait for the business to complete."
Note: The quantized constraint of a detailed requirement should be described in the subsection of the detailed requirement.
本子章节应从整体上描述静态和动态的量化的对软件(或人与软件交互)的需求。

静态的量化需求可能包括:

A. 支持的终端数目。

B. 支持的同时使用的用户数目。

C.处理的文件和记录的数目。

D.表和文件的大小。

动态的量化需求可能包括:

A. 在正常和峰值工作量条件下特定时间段(如一小时)

B. 处理的事务和任务的数目以及数据量。

所有的这些需求应以可测量的术语进行描述,例如所有的操作应在1秒内被处理完成,而不是描述成操作员不必等待操作的完成。

注意: 用于一个具体功能的量化限制通常在该功能的处理子章节中描述。
\section{外部接口需求}
\subsection{用户接口}
\label{ssec:ui}
<The interface of the system with the User and vice versa should be explained in detail. >

详细描述系统与用户之间的接口

This section should include:
A. Features that must be supported by the software for eachman-machine interface. For example, if the user operates from a display terminal, then the following should be included:
		Screen format required
		Page layout and content of report and menu
		Timing sequence for input and output
		Usage of some functional key combinations
B. Every aspect about the use of the system's user interface. It could be a list that shows the user what should do and what should not do.  For example, an option of overlong or overshort message. . And same as other requirements, these requirements should be easily verified. For example, saying "A level 4 typist can finish function X in Z minutes after a one-hour training." instead of "A typist can finish function X"	

这应描述下述内容:

A. 对每种人机界面,软件所必须支持的特性。例如,如果系统用户通过一个显示终端进行操作,那么应包含下述内容:
要求的屏幕格式
页面规划及报告或菜单的内容
输入和输出的相关时序
一些组合功能键的用法

B. 与系统用户接口使用相关的所有方面。这可能只是一个简单的关于系统怎样展示给用户而该做什么和不该做什么的列表。例如提供关于长或短错误消息选项。和所有其它需求一样,这些需求也应能被检验,例如,四级打字员经一小时的培训后能在Z分钟内完成功能X,而不是一个打字员能完成功能X。

\subsection{软件接口}
<The interface with other system/modules/projects should be explained in detail. >

详细描述与其他系统 /模块 /项目之间的接口

Describes how to use the other (required) software products. (such as data management system, operation system, or algorithm tools package), and the interfaces to other application systems (such as interfaces between the protocol process system and the database management system )
For each required software product, following information should be provided:
A. Name
B. Mnemonic symbol
C. Version number
D. Source
For each interface, this section should:
A. Discuss the objective of the required software.
B. Define the interfaces by content and format of message/function. If the interfaces have been clearly described in other documents, it is not necessary to describe in detail here. But the reference of those documents should be given.

在此应描述如何使用其它(必需的)软件产品(例如,数据管理系统,操作系统,或算法工具包),以及与其它应用系统的接口(例如,协议处理系统和数据库管理系统之间的接口)。

对每个必需的软件产品,应提供下列信息:
A.	名字
B.	助记符
C.	版本号
D.	来源

对每个接口,本部分应:

A .	讨论与本软件产品相关的接口软件的目的。

B.	按消息/函数内容和格式定义接口。如果接口已在其它文档中很清楚地描述,就没有必要在这儿进行详细描述,但需说明应参考的文档。

\subsection{硬件接口}
<The interface with other hardware components should be explained in detail. >

详细描述与硬件的接口

Describes the logical features of the interface between the software and hardware components, including the equipment supported and how the equipment and protocol is supported. 

Defines the interfaces according to the content and format of the software/hardware protocol. If the interfaces have been clearly described in other documents, it is not necessary to describe in detail here. But the reference of those documents should be given.

在此描述软件产品和系统硬件组件之间接口的逻辑特征,也包括支持哪些设备、怎样支持这些设备和协议等。
 
按软/硬件协议内容和格式定义接口。如果接口已在其它文档中很清楚地描述,就没有必要在这儿进行详细描述,但需说明应参考的文档。

\subsection{通讯接口}
<This should specify the various interfaces to communications such as local network protocols, etc.>

详细描述通讯接口,如本地网络协议等。

Defines the interfaces according to the content and format of the message/function. If the interfaces have been clearly described in other documents, it is not necessary to describe in detail here. But the reference of those documents should be given.

按消息/函数内容和格式定义接口。如果接口已在其它文档中很清楚地描述,就没有必要在这儿进行详细描述,但需说明应参考的文档。
