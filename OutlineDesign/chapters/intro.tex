%!TEX root = ../main.tex

\chapter{引言}
\section{编写目的}
在本项目的前一阶段,也就是需求分析阶段,已经将系统用户对本系统的需求做了详细的阐述,这些用户需求已经在上一阶段中对不同用户所提出的不同功能,实现的各种效果做了调研工作,并在需求规格说明书中得到详尽得叙述及阐明。

本阶段已在系统的需求分析的基础上,对
四季云音乐 (音乐播放系统)
做概要设计。主要解决了实现该系统需求的程序模块设计问题。包括如何把该系统划分成若干个模块、决定各个模块之间的接口、模块之间传递的信息,以及数据结构、模块结构的设计等。在以下的概要设计报告中将对在本阶段中对系统所做的所有概要设计进行详细的说明,在设计过程中起到了提纲挈领的作用。

在下一阶段的详细设计中,程序设计员可参考此概要设计报告,在概要设计即时聊天工具所做的模块结构设计的基础上,对系统进行详细设计。在以后的软件测试以及软件维护阶段也可参考此说明书,以便于了解在概要设计过程中所完成的各模块设计结构,或在修改时找出在本阶段设计的不足或错误。

\R {
    经过项目委员会的详细讨论,最终制订了一系列项目需求的更改与新增,在此基础上,我们对设计文档的内容进行了相应的修改。我们制定这些修改,并以红色字体标注这些变更,使得已经进行的项目实施过程在根据需求的修订而进行相应的修改的时候可以更好的对比各个需求点的变更,能够更加方便的进行需求分析与实现。
}


\section{项目背景}
随着媒体播放技术的不断发展,人们需要更好用、更方便地
音乐播放系统。
在传统的音乐软件中,软件设计者仅仅将功能点设计在
播放技术与音乐的元信息本地管理上。
移动互联网的到来,大大加速了音乐APP的发展,完全取代最早之前通过电脑下载歌曲到手机的内存卡中的时代。用户希望随时随地就能听到自己想要的音乐,不单单仅靠自己的搜索获取到这些歌曲,显然不能满足用户的需求。
随着云技术的发展,各种各样的数据可以在
不同的平台与设备之间互通,
而用户收听音乐的场景也是多种多样的,
比如,手机、电脑、车载娱乐系统、电视机顶盒,
所以,我们制作一款云音乐播放系统,
帮助了用户在不同的场景下都可以得到
一致的音乐服务体验。

通过歌单作为入口发现音乐,用户可以根据自己的音乐口味喜好来获取优质音乐,既优化了用户体验,提高用户粘性,也方便收集用户数据,为迭代和运营策略做有力的依据。同时也增强了用户间的互动,体现社交属性,促进产生更多优质的UGC
(User Generated Content,用户原创内容)
内容。

我们的项目同时具有可盈利性,
因为人们已经逐渐开始具有版权意识,
并愿意为数字产品付费,
我们注重为用户提供高品质、低延迟、云体验的
音乐服务,并通过音乐付费、音乐订阅来盈利。

\section{术语}
[列出本文档中所用到的专门术语的定义和外文缩写的原词组]
\begin{table}[htbp]
\centering
\caption{术语表} \label{tab:terminology}
\begin{tabular}{|c|c|}
    \hline
    \hline
    术语 & 中文解释 \\
    \hline
    Application Programming Interface & 应用程序接口\\
    \hline
    User Interface & 用户接口\\
    \hline
    Hypertext Transfer Protocol & 超文本传输协议\\
    \hline
    High Frequency Low Size & 高频小尺寸\\
    \hline
    Low Frequency High Size & 低频大尺寸\\
    \hline
\end{tabular}
% \note{这里是表的注释}
\end{table}