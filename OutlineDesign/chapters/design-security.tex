\chapter{安全保密设计}

\section{通信安全}

客户与服务器的连接全部使用HTTPS进行,保证了服务器和用户浏览器之间的传输内容是可信的,
同时防止了用户个人信息在传输过程中被窃听。

\section{服务端安全}

\subsection{服务端入侵防护}

服务器只允许SSH登录,并且合理设计权限管理,使得重要信息只能被管理员读取。
同时,使用了Nginx反向代理了Python监听的端口。运行了Fail2ban等软件,
屏蔽对服务器的恶意访问,保障了服务器能持续稳定为用户提供服务。

\subsection{数据库安全}

我们的后端数据库是PostegreSQL,通过合理设置权限,只有管理员和应用程序账号
能修改表的数据和读取用户隐私相关的数据。同时,数据库进行定时热备份,保证了数据库
即使在服务器出问题丢失数据的情况下迅速恢复。

\subsection{用户认证信息安全}

我们使用用户名和密码来进行身份认证。在服务器中,我们不原文存储密码,而是对密码进行
加盐后哈希,这种方式能抵抗穷举攻击和彩虹表攻击等常见攻击方式,能保证即使密码数据库外泄,也无法还原得到原始的密码,大大提升了系统的安全性。