\chapter{出错处理设计}
\section{数据库出错处理}

数据库备份依据数据库的不同类型与不同特征分别采取不同的备份方案。

用户等数据改变频率比较低,改变量比较小,采用事务日志备份。由于该数据库与用户登录的核心功能关联,该数据库出错时,如果有充足的服务条件,先切换到有最后一次正确的数据库的服务器上继续提供服务,等主服务器将备份数据还原之后再切换回主服务器提供服务,无需暂停服务。

评论数据等,每次改变量比较小,但改变频率极为频繁,因此采用增量和定时备份的方式备份。由于该数据库并未与核心功能关联,该数据库出错时,如果有充足的服务器条件,由备用服务器提供该服务,而不需暂停服务,主服务器仍旧提供其余部分服务,恢复备份数据后移回主服务器提供服务。

而文件数据(歌曲,图片)由于大部分情况为单个文件的变化,所以采用完全备份。该数据库出错时,如果有充足的服务器条件,可以先排查出现错误的具体功能,由备用服务器提供该功能的服务,不需暂停服务,主服务器仍旧提供其余部分服务,恢复备份数据后移回主服务器提供服务。

\section{某模块失效处理}
模块失效时,根据失效模块的功能和重要性决定处理方式。

若核心模块如储存、用户、数据库等模块失效,则应暂停整个系统的服务,由该模块的开发人员为主,在其他开发人员的支持下调整失效模块,测试完成后整体上线。

若非核心模块如图形界面、审计模块、网络请求等失效,则应先关闭失效模块提供的服务及相应接口提供的服务,其余不相关的服务维持其状态,由该模块的开发人员为主,在其他开发人员的支持下调整失效模块,测试完成后将更改的模块同时上线。