\chapter{任务概述}
本系统的目标是实现一个云音乐储存、管理、播放、运营系统,包括客户端、服务器端两个部分。

服务端面向企业管理,可以用于管理服务器中的音乐文件,并且同时存储所有用户的个人信息、
账户信息、购买信息以及用于用户体验提升的个人音乐喜好信息。

客户端包含多种平台与形式,包括:网页Web客户端、桌面应用客户端、移动设备客户端,
其中,桌面设备客户端包括Windows、Mac OS、Linux三个平台、 而移动设备客户端包括iOS、
Android等客户端, 并且, 我们还需要为客户合作的智能电视合作和车载娱乐设备做适配。

客户端面向普通用户,用户可以在客户端上收听海量音乐,音乐的数据由服务端提供。
用户可以在各个平台上同步自己的歌单、收藏、账户信息、购买信息,并且可以评论、收藏、
评价、分享平台上的所有歌曲。

\section{目标}
实现云音乐系统,实现需求规格说明书中所描述的网络音乐文件传输功能、音乐流播放功能和
账户管理及支付管理功能,并且保证系统的健壮性和数据安全。

\newpage
\section{开发与运行环境}

\subsection{开发环境的配置}
\begin{table}[h]
\centering
\caption{开发环境的配置} \label{tab:development-environment}
\begin{tabular}{|c|c|c|}
    \hline
    类别 & 标准配置 & 最低配置 \\
    \hline
    计算机硬件 & \tabincell{c}{基于x86结构的CPU\\ 主频>=3.2GHz\\ 内存>=16G\\ 硬盘>=1TB} & \tabincell{c}{基于x86结构的CPU\\ 主频>=2.0GHz\\ 内存>=4GB\\ 硬盘>=512G} \\
    \hline
    计算机软件 & 
    \tabincell{c} {
        Linux (kernel version>=4.10)\\ 
        GNU gcc (version>=6.3.1)\\
        Qt Developer Kit (version>=5.30)\\
        Python (version>=2.7)\\
        Web Storm (version>=8.0)\\
        Android Studio (可选)\\
        } & 
    \tabincell{c} {
        Linux (kernel version>=3.10)\\ 
        GNU gcc (version>=5.4)\\
        Qt Developer Kit (version>=5.10)\\
        Python (version>=2.7)\\
        Web Storm (version>=8.0)\\
        } \\
    \hline
    网络通信 & \tabincell{c}{至少要有一块可用网卡\\ 能运行IP协议栈即可} & \tabincell{c}{至少要有一块可用网卡\\ 能运行IP协议栈即可} \\
    \hline
    其他 & 采用 postgresql 数据库 & 采用 postgresql 数据库 \\
    \hline
\end{tabular}
~\\~备注:iOS 应用的开发需要使用 Apple Macbook Pro 
笔记本电脑产品,或者 Apple Mac Pro 系列产品。
\end{table}

\newpage
\subsection{测试环境的配置}
\begin{table}[htbp]
\centering
\caption{PC测试环境的配置} \label{tab:pctest-environment}
\begin{tabular}{|c|c|c|}
    \hline
    类别 & 标准配置 & 最低配置 \\
    \hline
    计算机硬件 & \tabincell{c}{基于x86结构的CPU\\ 主频>=3.2GHz\\ 内存>=16G\\ 硬盘>=1TB} & \tabincell{c}{基于x86结构的CPU\\ 主频>=2.0GHz\\ 内存>=4GB\\ 硬盘>=512G} \\
    \hline
    计算机软件 & 
    \tabincell{c} {
        Linux (kernel version>=4.10)\\ 
        GNU gcc (version>=6.3.1)\\
        Qt Runtime (version>=5.30)\\
        Python (version>=2.7)\\
        } & 
    \tabincell{c} {
        Linux (kernel version>=3.10)\\ 
        GNU gcc (version>=5.4)\\
        Qt Runtime (version>=5.10)\\
        Python (version>=2.7)\\
        } \\
    \hline
    网络通信 & \tabincell{c}{至少要有一块可用网卡\\ 能运行IP协议栈即可} & \tabincell{c}{至少要有一块可用网卡\\ 能运行IP协议栈即可} \\
    \hline
    其他 & 采用 postgresql 数据库 & 采用 postgresql 数据库 \\
    \hline
\end{tabular}
% \note{这里是表的注释}
\end{table}

\begin{table}[htbp]
\centering
\caption{Android测试环境的配置} \label{tab:iostest-environment}
\begin{tabular}{|c|c|c|}
    \hline
    类别 & 标准配置 & 最低配置 \\
    \hline
    硬件设备 & \tabincell{c}{
        主流配置的安卓系统手机设备\\
        尽量满足多种分辨率
    } & \tabincell{c}{
        标准配置的部分机型\\
        至少满足所有主要\\
        分辨率的机型各一部
    } \\
    \hline
        手机硬件 & 
        \tabincell{c}{
            基于ARM结构的CPU\\ 
            主频>=2.4GHz\\ 
            内存>=4G\\ 
            硬盘>=128G} & 
        \tabincell{c}{
            基于ARM结构的CPU\\ 
            主频>=1.6GHz\\ 
            内存>=2G\\ 
            硬盘>=32G} \\
    \hline
\end{tabular}
% \note{这里是表的注释}
\end{table}

\begin{table}[htbp]
\centering
\caption{iOS测试环境的配置} \label{tab:iostest-environment}
\begin{tabular}{|c|c|c|}
    \hline
    类别 & 标准配置 & 最低配置 \\
    \hline
    硬件设备 & \tabincell{c}{
        iPhone 6 测试开发机\\
        iPhone 6 Plus 测试开发机\\
        iPhone 6S 测试开发机\\
        iPhone 6S Plus测试开发机\\
        iPhone 7 Plus 测试开发机\\
        iPhone 7S 测试开发机\\
        iPhone 7S Plus测试开发机\\
        iPhone 8 测试开发机\\
        iPhone 8 Plus测试开发机\\
        iPhone 8S 测试开发机\\
        iPhone 8S Plus测试开发机\\
        iPhone X 测试开发机
    } & \tabincell{c}{
        标准配置的部分机型\\
        至少满足所有分辨率的机型各一部
    } \\
    \hline
        手机硬件 & 
        \tabincell{c}{
            由于iPhone硬件较统一\\不做详细限定\\
            存储容量建议满足>=64GB
        } &
        \tabincell{c}{
           由于iPhone硬件较统一\\不做详细限定\\
            存储容量建议至少>=16GB
        } \\
    \hline
\end{tabular}
% \note{这里是表的注释}
\end{table}

\newpage
\subsection{运行环境的配置}

\begin{table}[htbp]
\centering
\caption{PC运行环境的配置} 
\begin{tabular}{|c|c|c|}
    \hline
    类别 & 标准配置 & 最低配置 \\
    \hline
    计算机硬件 & \tabincell{c}{基于x86结构的CPU\\ 主频>=2.4GHz\\ 内存>=2G\\ 硬盘>=256G} & \tabincell{c}{基于x86结构的CPU\\ 主频>=1.6GHz\\ 内存>=512M\\ 硬盘>=32G} \\
    \hline
    计算机软件 & 
    \tabincell{c} {
        Linux (kernel version>=4.10)\\ 
        或\\
        Windows 7 以上\\
        Qt Runtime (version>=5.30)\\
        } & 
    \tabincell{c} {
        Linux (kernel version>=3.10)\\ 
                或\\
        Windows 7 以上\\
        Qt Runtime (version>=5.10)\\
        } \\
    \hline
    网络通信 & \tabincell{c}{至少要有一块可用网卡\\ 能运行IP协议栈即可} & \tabincell{c}{至少要有一块可用网卡\\ 能运行IP协议栈即可} \\
    \hline
    其他 & 采用 postgresql 数据库 & 采用 postgresql 数据库 \\
    \hline
\end{tabular}
% \note{这里是表的注释}
\end{table}

\begin{table}[htbp]
\centering
\caption{Android运行环境的配置} \label{tab:iostest-environment}
\begin{tabular}{|c|c|c|}
    \hline
    类别 & 标准配置 & 最低配置 \\
    \hline
    硬件设备 & \tabincell{c}{
        3年内主流配置以上的\\
        安卓系统手机设备
    } & \tabincell{c}{
        6年内主流配置以上的\\
        安卓系统手机设备
    } \\
    \hline
        手机硬件 & 
        \tabincell{c}{
            基于ARM结构的CPU\\ 
            主频>=2.2GHz\\ 
            内存>=2G\\ 
            硬盘>=64G} & 
        \tabincell{c}{
            基于ARM结构的CPU\\ 
            主频>=1.5GHz\\ 
            内存>=1G\\ 
            硬盘>=16G} \\
    \hline
\end{tabular}
% \note{这里是表的注释}
\end{table}

\begin{table}[htbp]
\centering
\caption{iOS测试环境的配置} \label{tab:iostest-environment}
\begin{tabular}{|c|c|c|}
    \hline
    类别 & 标准配置 & 最低配置 \\
    \hline
    硬件设备 & \tabincell{c}{
        iPhone 6 测试开发机\\
        iPhone 6 Plus 测试开发机\\
        iPhone 6S 测试开发机\\
        iPhone 6S Plus测试开发机\\
        iPhone 7 Plus 测试开发机\\
        iPhone 7S 测试开发机\\
        iPhone 7S Plus测试开发机\\
        iPhone 8 测试开发机\\
        iPhone 8 Plus测试开发机\\
        iPhone 8S 测试开发机\\
        iPhone 8S Plus测试开发机\\
        iPhone X 测试开发机
    } & \tabincell{c}{
        标准配置的部分机型\\
        至少满足所有分辨率的机型各一部
    } \\
    \hline
        手机硬件 & 
        \tabincell{c}{
            由于iPhone硬件较统一\\不做详细限定\\
            存储容量建议满足>=64GB
        } &
        \tabincell{c}{
           由于iPhone硬件较统一\\不做详细限定\\
            存储容量建议至少>=16GB
        } \\
    \hline
\end{tabular}
% \note{这里是表的注释}
\end{table}

\newpage

\section{需求概述}

我们的项目分为服务端与客户端,其中,服务端是统一的,所有客户端都由
该统一服务端进行数据交互与信息支持;而为了满足用户多种多样的收听音乐
的环境,我们的客户端有较多种类。

\subsection{服务端功能需求} 

在服务器端,我们需要可靠的数据库来支持音乐服务的运行,这些数据库储
存必要的信息,同需求报告中所述,我们需要:
    \begin{enumerate}
        \item \textbf{音乐数据库},储存了各种音质版本的音乐数据文件;
        \item \textbf{音乐元数据(Meta Data)数据库},储存所有音乐的相关信息,如
            作者、所属专辑、发行日期等;
        \item \textbf{商品数据库},需要储存所有音乐的价格以及优惠信息等;
        \item \textbf{用户数据库},需要储存所有用户的信息,包括用户名、密码、密保信息、
            订阅信息、账户余额等;
        \item \textbf{用户-商品库},存储用户对音乐的购买信息,包括用户的购物车信息、
            已购买音乐的支付信息与历史记录等;
        \item \textbf{用户-音乐库},存储用户与音乐之间的联系,
            具体而言,储存了用户对听过的音乐的喜爱程度、评分、发出的评论,
            并且储存了用户收听音乐的历史信息,如在何时,听了某首音乐多久、循环次数等。
    \end{enumerate}

服务器端不仅需要数据库服务来存储软件的数据,并且,还需要相关的接口,
各种客户端通过调用这些服务接口来运行功能,
具体而言,我们需要以下的服务器接口:
\begin{enumerate}
    \item \textbf{音乐下载接口},通过该接口,客户端可以获取需要的音乐文件,
        我们需要保证在尽可能多的时间段内,用户都可以获取稳定、高速的下载服务;
    \item \textbf{流播放服务接口},该接口提供了音乐的串流服务,主要用于在线收听没有下载
        的音乐,根据流数据服务的时效性与响应紧急性,我们需要每一个客户端的
        流服务请求都能得到尽可能快的相应;
    \item \textbf{用户信息交互接口},该接口提供了用户登陆信息的传递以及验证的功能
        (用户注册功能亦然),我们希望该服务可以长期稳定;
    \item \textbf{商品交易服务接口},该接口提供了从客户端购买音乐的功能的基础,
        当用户发起支付的需求时,对于内部支付方式(账户余额)以及多种支付方式
        (信用卡、支付宝、微信支付以及未来可能出现的各种支付方式)都应该
        可以正确、稳定、高效的处理;
\end{enumerate}

\subsection{客户端功能需求}

 作为一款音乐播放软件,我们首先要满足一系列基础功能,这些功能在所有
的客户端、所有的平台上都能够完整地被实现,并且都应有统一的界面。
我们需要实现以下界面:
\begin{enumerate}
    \item \textbf{用户主界面}:
    该界面可以进入一系列音乐集 。
        在该界面,可以看到
        ``本地音乐''、``我的最爱''、``最近播放''
        这三个默认的入口。
        在默认音乐集下面,有所有用户创建的音乐集的入口列表,
        分类为``我的歌单'',并将所有的用户自创歌单按照用户指定的顺序展示。
    \item \textbf{音乐集查看界面}:
    进入一个音乐集之后,将展示该音乐集的信息和功能按键,应包括:
    音乐集封面、音乐集名称、音乐集创建者、功能按键栏、音乐集内的所有音乐;
    并且,音乐集界面内,应包含返回按钮,来退回进入音乐集之前的界面。
    \item \textbf{音乐播放界面}:
    从搜索功能、音乐集内、通过推荐或通过外部链接可以进入音乐播放界面,
        在该界面中,可以控制音乐的播放,以及查看与该音乐有关的信息,
    \item \textbf{音乐推荐界面}:
        该界面是产品的核心功能之一,该界面主要分为两部分:
        \begin{itemize}
            \item 个性推荐模块:
                该模块中,主要根据该用户自身的历史信息以及主观喜好,
                来推荐音乐。
            \item 热门推荐模块:
                该模块中,主要根据全网用户数据来推荐音乐,
                以及展示排行榜。
        \end{itemize}
    \item \textbf{账户界面}:
        在该界面中,若用户没有登陆,则应该显示登陆与注册的功能,
        若已登录,则可以查看账户的相关信息或进行相关操作。
    \item \textbf{设置界面}:
        在该界面,我们需要能够对软件行为作出一定的设置,该设置需要针对不同的平台与使用场景来展示不同的设置选项。
\end{enumerate}

\section{条件与限制}

\subsection{设计条件与限制} % (fold)

在设计上,我们需要满足以下的设计规范。

        \begin{itemize}
            \item HTML代码符合W3C-HTML5编码规范[\cite{hickson2011html5}]
            \item XML配置符合W3C-XML编码规范 [\cite{bray1997extensible}]
            \item JSON数据包符合ECMA-404标准 [\cite{bray2017javascript}]
            \item 后端开发符合PEP8 Python编码规范 [\cite{van2001pep}]
            \item UI设计符合Material Design规范
        \end{itemize}

\subsection{设计条件与限制} % (fold)

    由于我们在开发过程中的软件使用计划,以及使用的
    网络、图形界面运行库的选择,我们的产品的服务端需要满足
    以下软件环境约束的环境中运行:
    \begin{itemize}
        \item 数据库: postgresql
        \item 后端开发: Python
        \item 并行操作: 支持
        \item 通信协议: HTTP/HTTPS[\cite{berners1996hypertext}][\cite{fielding1999hypertext}][\cite{rescorla2000http}]
    \end{itemize}

    而我们的产品的客户端需要在以下的软件环境约束中运行:
    \begin{itemize}
    \item HTML: HTML5标准[\cite{hickson2011html5}]
    \item 前端框架: Vue
    \item CSS版本: CSS3
    \item 样式框架: Bootstrap 4.0
    \item 通信协议: HTTP/HTTPS[\cite{berners1996hypertext}][\cite{fielding1999hypertext}][\cite{rescorla2000http}]
    \item 音乐格式: MEPG-3[\cite{le1991mpeg}]
\end{itemize}

\subsection{硬件条件与限制} % (fold)

        为了满足现代软件需要的图形效果、网络服务,我们的产品需要满足
        以下硬件约束的环境中运行:
        \begin{itemize}
            \item Web端: 1GB以上内存
            \item iOS端: 安装了iOS 9及以上的iPhone或iPad
            \item Android端: 安装了Android 6.0及以上的设备,1GB以上内存
            \item 服务端: Debian 9操作系统,E5处理器,4GB内存,1Gbps及以上网络接入
        \end{itemize}

