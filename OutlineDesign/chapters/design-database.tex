\chapter{数据库设计}
\section{数据库环境说明}
本系统的数据系统采用PostgreSQL数据库系统。


\section{数据库的命名规则}
\begin{itemize}
	\item 不允许使用缩写
	\item 表名使用单数形式。对于有关联的表,属性名用表名+id的方式来标明这是外表的一个主键
	\item 字段名字不带前缀
\end{itemize}

\section{逻辑设计}

数据模型满足3NF\\
\newpage

\begin{figure}[ht]
	\centering
	\includegraphics[width=14cm]{database_ER}
	\caption{ER模型图}\label{fig:ER}
\end{figure}

\section{物理设计}

\subsection{数据库产品}

数据库采用了PostgreSQL,由于暂时数据访问量不大,所以不使用分布式结构。

\subsection{实体属性、类型、精度}

\subsubsection{用户数据表设计}

用户数据表保存了用户的基本信息

\begin{table}[htbp]
\centering
\caption{用户数据表Users设计} \label{tab:users-database}
\begin{tabular}{|c|c|c|c|c|}
    \hline
    字段名 & 类型 & 大小 & 说明 & 备注 \\
    \hline
    ID & char & 64 & 用户的唯一标识符 & 主键\\
    \hline
    password & char & 256 & 用户密码的哈希 & · \\
    \hline
    salt & char & 256 & 盐 & · \\
    \hline
    phone & char & 18 & 电话号码 & 外键 \\
    \hline
    email & char & 32 & 电子邮箱 & · \\
    \hline
    description & char & 1024 & 个人自述 & · \\
    \hline
    avatar\_url & char & 256 & 头像地址 & · \\
    \hline
    last\_login & timestamp & 32 & 上次登录时间 & · \\
    \hline
    music\_lists & json & var & 自定义歌单的集合 & . \\
    \hline
\end{tabular}
\note{用户数据表Users设计}
\end{table}

\subsubsection{电话用户对应关系表PhoneToUser设计}

电话用户对应关系表保存了用户电话到用户ID的对应关系

\begin{table}[htbp]
\centering
\caption{电话用户对应关系表PhoneToUser设计} \label{tab:phone-user-database}
\begin{tabular}{|c|c|c|c|c|}
    \hline
    字段名 & 类型 & 大小 & 说明 & 备注 \\
    \hline
    phone & char & 18 & 用户电话号码 & 主键\\
    \hline
    user\_id & char & 64 & 对应用户 & 外键,来自表Users \\
    \hline
\end{tabular}
\note{电话用户对应关系表PhoneToUser设计}
\end{table}

\subsubsection{音乐信息数据库Musics设计}

音乐信息数据库保存了音乐的信息

\begin{table}[htbp]
	\centering
	\caption{音乐信息数据库Musics设计} \label{tab:music-database}
	\begin{tabular}{|c|c|c|c|c|}
		\hline
		字段名 & 类型 & 大小 & 说明 & 备注 \\
		\hline
		ID & char & 64 & 音乐唯一标识符 & 主键\\
		\hline
		name & char & 64 & 音乐名称 & · \\
		\hline
		author & char & 64 & 作者 & · \\
		\hline
		length & smallint & 16 & 时长(秒) & · \\
		\hline
		size & int & 32 & 大小(字节) & · \\
		\hline
	\end{tabular}
	\note{音乐信息数据库Musics设计}
\end{table}

\subsubsection{订阅信息数据库Subscriptions设计}

订阅信息数据库保存了订阅的信息

\begin{table}[htbp]
	\centering
	\caption{订阅信息数据库Subscriptions设计} \label{tab:subscription-database}
	\begin{tabular}{|c|c|c|c|c|}
		\hline
		字段名 & 类型 & 大小 & 说明 & 备注 \\
		\hline
		ID & char & 64 & 订阅唯一标识符 & 主键\\
		\hline
		name & char & 64 & 订阅名称 & · \\
		\hline
		contents & json & var & 包含的音乐的集合  & · \\
		\hline
	\end{tabular}
	\note{订阅信息数据库Subscriptions设计}
\end{table}

\subsubsection{用户音乐购买信息MusicSells设计}

用户音乐购买信息保存了相当于订阅信息倒排索引的信息

\begin{table}[htbp]
	\centering
	\caption{用户音乐购买信息MusicSells设计} \label{tab:music-sell-database}
	\begin{tabular}{|c|c|c|c|c|}
		\hline
		字段名 & 类型 & 大小 & 说明 & 备注 \\
		\hline
		user\_ID & char & 64 & 用户ID & 主键\\
		\hline
		music\_ID & char & 64 & 音乐ID & 主键 \\
		\hline
		count & smallint & 8 & 累计次数  & · \\
		\hline
	\end{tabular}
	\note{用户音乐购买信息MusicSells设计}
\end{table}

\subsubsection{自定义歌单数据库MusicList设计}

自定义歌单数据库保存了歌单的信息

\begin{table}[htbp]
	\centering
	\caption{自定义歌单数据库MusicList设计} \label{tab:music-list-database}
	\begin{tabular}{|c|c|c|c|c|}
		\hline
		字段名 & 类型 & 大小 & 说明 & 备注 \\
		\hline
		ID & char & 64 & 歌单唯一标识符 & 主键\\
		\hline
		name & char & 64 & 歌单名称 & · \\
		\hline
		contents & json & var & 包含的音乐的集合  & · \\
		\hline
	\end{tabular}
	\note{自定义歌单数据库MusicList设计}
\end{table}

\subsubsection{用户评论Comments设计}

自定义歌单数据库保存了歌单的信息

\begin{table}[htbp]
	\centering
	\caption{用户评论Comments设计} \label{tab:music-list-database}
	\begin{tabular}{|c|c|c|c|c|}
		\hline
		字段名 & 类型 & 大小 & 说明 & 备注 \\
		\hline
		ID & char & 64 & 用户评论唯一标识符 & 主键 \\
		\hline
		UserID & char & 64 & 用户ID & 外键 \\
		\hline
		MusicID & char & 64 & 歌曲ID & 外键 \\
		\hline
		comment & char & 512 & 用户对歌曲的评论  & · \\
		\hline
	\end{tabular}
	\note{用户评论Comments设计}
\end{table}

\section{安全性设计}	
通过合理设置权限,只有管理员和应用程序账号
能修改表的数据和读取用户隐私相关的数据。同时,数据库进行定时热备份,保证了数据库
即使在服务器出问题丢失数据的情况下迅速恢复。

\section{数据库管理与维护说明}
监视数据库进程,如果出现无法连接马上告警。

定时更新数据库安全补丁。