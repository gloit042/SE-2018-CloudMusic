\chapter{数据结构设计}
\section{逻辑结构设计}
\subsection{用户数据结构}

class User:             用户类\\
\indent \indent    UID : int           每个账户的唯一标识符\\
\indent \indent    Username: String    账户用户名\\
\indent \indent    Password: String    账户密码\\
\indent \indent    AccessToken: String 每次登陆生成的 Token

\subsection{歌曲数据结构}

class Song:\\
\indent \indent   SID : int           每首歌曲的唯一标识符\\
\indent \indent   Name : String       歌曲名\\
\indent \indent   Author : String list 歌曲的作者名\\
\indent \indent   AID : int           歌曲的专辑标识符\\
\indent \indent   Available : bool    歌曲是否已下架\\
\\
class Album:\\
\indent \indent    AID : int           专辑的标识符\\
\indent \indent    Name : String       专辑的名称\\
\indent \indent    Songs : int list list   顺序排列的歌曲曲目,用标识符表示\\
\indent \indent                            (每一列表示一张碟)
\indent \indent    Release : Datetime  专辑公开日期

\subsection{购买情况数据结构}

class PurchaseSong:\\
\indent \indent    SID : int           歌曲的标识符\\
\indent \indent    UID : int           用户的标识符\\
\indent \indent    ListenAvailable : bool    用户是否能够听该曲\\
\indent \indent    DownloadAvailable : bool  用户能否下载该曲\\
\indent \indent    ListenUntil : Datetime    听歌的有效期限\\
\indent \indent    DownloadUntil : Datetime  下载的有效期限

\subsection{评论数据结构}

class Comment:\\
\indent \indent    SID : int           评论所在歌曲的标识符\\
\indent \indent    UID : int           评论者的标识符\\
\indent \indent    Text : String       评论的内容\\
\indent \indent    Date : Datetime     评论的时间

\subsection{本地缓存数据结构}

class CacheState:       本地的缓存状况\\
\indent \indent    type : enum{Music, AlbumImage, UserImage}
\indent \indent                        缓存内容的种类\\
\indent \indent    ID : int            资源的标识符\\
\indent \indent    Ready : bool        资源是否已下载到本地\\
\indent \indent    Size : int          缓存资源的大小\\
\indent \indent    Downloaded : int    已下载部分的大小

\section{物理结构设计}
各数据结构无特殊物理结构要求。

\section{数据结构与程序模块的关系}
\begin{table}[htbp]
\centering
\caption{客户端数据结构与程序代码的关系表} \label{tab:datastructure-module}
\begin{tabular}{|c|c|c|c|c|}
    \hline
    . & 听歌模块 & 购买模块 & 社交模块 & 下载模块 \\
    \hline
    用户数据结构 & Y & Y & Y & Y \\
    \hline
    评论数据结构 & . & . & Y & . \\
    \hline
    歌曲数据结构 & Y & . & . & Y \\
    \hline
    购买情况数据结构 & . & Y & . & Y \\
    \hline
    本地缓存数据结构 & Y & . & Y & Y \\
    \hline
\end{tabular}
\note{客户端各项数据结构的实现与各个程序模块的分配关系}
\end{table}

\begin{table}[htbp]
    \centering
    \caption{服务端数据结构与程序代码的关系表} \label{tab:datastructure-module}
    \begin{tabular}{|c|c|c|c|c|}
        \hline
        . & 听歌模块 & 购买模块 & 社交模块 & 下载模块 \\
        \hline
        用户数据结构 & Y & Y & Y & Y \\
        \hline
        评论数据结构 & . & . & Y & . \\
        \hline
        歌曲数据结构 & Y & . & . & Y \\
        \hline
        购买情况数据结构 & . & Y & . & Y \\
        \hline
        本地缓存数据结构 & . & . & . & . \\
        \hline
    \end{tabular}
    \note{服务器端各项数据结构的实现与各个程序模块的分配关系}
    \end{table}