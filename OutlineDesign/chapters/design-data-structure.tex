\chapter{数据结构设计}
\section{逻辑结构设计}
\subsection{用户数据结构}

class User:             用户类\\
\indent \indent    UID : int           每个账户的唯一标识符\\
\indent \indent    Username: String    账户用户名\\
\indent \indent    Password: String    账户密码\\
\indent \indent    AccessToken: String 每次登陆生成的 Token

\R{
\subsection{歌曲数据结构}
class Qual:\\
\indent \indent   Channel : int       歌曲的声道数\\
\indent \indent   SampleRate : int    歌曲的采样率\\
\indent \indent   Depth : int         歌曲的采样深度\\
}

class Song:\\
\indent \indent   SID : int           每首歌曲的唯一标识符\\
\indent \indent   Name : String       歌曲名\\
\indent \indent   Author : String list 歌曲的作者名\\
\indent \indent   AID : int           歌曲的专辑标识符\\
\indent \indent   Available : bool    歌曲是否已下架\\
\indent \indent   Score : float       歌曲评分\\
\R{
\indent \indent   Quality : Qual list 可用的歌曲质量列表\\
}
\indent \indent   Path : path \R{list}  \R{不同质量的}歌曲在服务器上的路径\\
\\
class Album:\\
\indent \indent    AID : int           专辑的标识符\\
\indent \indent    Name : String       专辑的名称\\
\indent \indent    Songs : int list list   顺序排列的歌曲曲目,用标识符表示\\
\indent \indent                            (每一列表示一张碟)
\indent \indent    Release : Datetime  专辑公开日期

\subsection{购买情况数据结构}

class PurchaseSong:\\
\indent \indent    SID : int           歌曲的标识符\\
\indent \indent    UID : int           用户的标识符\\
\indent \indent    ListenAvailable : bool \R{list}   用户是否能够听\R{不同质量的}该曲\\
\indent \indent    DownloadAvailable : bool \R{list} 用户能否下载\R{不同质量的}该曲\\
\indent \indent    ListenUntil : Datetime \R{list}   听不同质量的歌的有效期限\\
\indent \indent    DownloadUntil : Datetime \R{list} 下载\R{不同质量的歌}的有效期限

\subsection{评论数据结构}

class Comment:\\
\indent \indent    SID : int           评论所在歌曲的标识符\\
\indent \indent    UID : int           评论者的标识符\\
\indent \indent    Text : String       评论的内容\\
\indent \indent    Date : Datetime     评论的时间

\subsection{本地缓存数据结构}

class CacheState:       本地的缓存状况\\
\indent \indent    type : enum{Music, \R{Image, Audio, Video}}
\indent \indent                        缓存内容的种类\\
\indent \indent    ID : int            资源的标识符\\
\indent \indent    Ready : bool        资源是否已下载到本地\\
\indent \indent    Size : int          缓存资源的大小\\
\indent \indent    Downloaded : int    已下载部分的大小\\
\indent \indent    File : path         缓存在磁盘的路径\\

\R{
\subsection{空间动态数据结构}
class Twit:
\indent \indent    ID : String         动态的ID\\
\indent \indent    Time : Datetime     动态的时间\\
\indent \indent    UID : int           发出动态的用户\\
\indent \indent    Content : String    动态的内容\\
\indent \indent    Up : int list       点赞的用户列表\\
}

\R{
\subsection{K歌数据结构}
class Cover:
\indent \indent    ID : int          歌曲的 ID\\
\indent \indent    User : int        用户的 ID\\
\indent \indent    Time : Datetime   唱歌的时间\\
\indent \indent    Audio : path      唱歌录音文件的路径\\
}


\section{物理结构设计}
各数据结构无特殊物理结构要求。

\section{数据结构与程序模块的关系}
\begin{table}[htbp]
\centering
\caption{客户端数据结构与程序代码的关系表} \label{tab:datastructure-module-client}
\begin{tabular}{|c|c|c|c|c|c|c|c|}
    \hline
    模块$\backslash$ 数据结构 & 用户 & 评论 & 歌曲 & 购买情况 & 本地缓存 & \R{空间动态} & \R{K歌}\\
    \hline
    储存 & Y & Y & Y & Y & Y & Y & Y \\
    \hline
    数据库 & Y & Y & Y & Y & . & . & .\\
    \hline
    用户    & Y & . & . & . & . & Y & Y \\
    \hline
    图形界面 & Y & Y & Y & Y & Y & Y & .\\
    \hline
    数据传输 & . & . & . & . & Y & Y & .\\
    \hline
    审计    & . & . & . & . & . & . & .\\
    \hline
    网络请求 & Y & Y & Y & Y & Y & Y & .\\
    \hline
\end{tabular}
\note{客户端各项数据结构的实现与各个程序模块的分配关系}
\end{table}

\begin{table}[htbp]
    \centering
    \caption{服务端数据结构与程序代码的关系表} \label{tab:datastructure-module-server}
    \begin{tabular}{|c|c|c|c|c|c|c|c|}
        \hline
        模块$\backslash$ 数据结构 & 用户 & 评论 & 歌曲 & 购买情况 & 本地缓存 & \R{空间动态} & \R{K歌}\\
        \hline
        储存 & Y & Y & Y & Y & . & Y & Y\\
        \hline
        数据库 & Y & Y & Y & Y & . & Y & .\\
        \hline
        用户 & Y & . & . & . & . & Y & Y\\
        \hline
        图形界面 & . & . & . & . & . & . & .\\
        \hline
        数据传输 & . & . & Y & . & . & Y & Y\\
        \hline
        审计 & Y & . & . & Y & . & . & .\\
        \hline
        网络请求 & Y & Y & Y & Y & . & Y & Y\\
        \hline
    \end{tabular}
    \note{服务器端各项数据结构的实现与各个程序模块的分配关系}
    \end{table}