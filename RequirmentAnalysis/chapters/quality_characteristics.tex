\chapter{软件质量特性}

\section {适应性}
\proname 的网页应可以在桌面和移动平台上访问,适配几乎所有平台的所有浏览器。\proname 的应用应可以在各种版本系统的电脑和手机上运行。\proname 的应用应支持 Android 系统的车载娱乐系统和机顶盒。

\section {可用性}
\proname 的网页客户端应具有简洁明了的UI,电脑和手机客户端应提供直观的 UI 和尽可能简单的菜单,避免菜单的多层嵌套和过多的菜单按键,使得其使用便利快捷。

\section {易学性}
\proname 的网页客户端应简洁明了,使得任何具有基本浏览网页技能的用户都可以轻松学习使用。 \proname 的客户端可以附有对首次使用的用户的指引。

\section {正确性}
\proname 的服务器端应对客户端的请求使用数据包校验和用户信息的比对,拒绝非法的请求。
\proname 的客户端在做出对用户账户的获取和变动之前,应与服务器通信比对用户凭据是否有效。

\section {灵活性}
\proname 的客户端应减少各个模块间的依赖,保证逻辑的简洁,并提供简单、全面的接口。使得开发和维护过程中可以根据需求的变更快速更新。

\section {可维护性}
\proname 的代码应有良好的文档和丰富的注释(占总代码的至少 30\%),加上简洁的逻辑,使得整个服务易于维护。

\section {可移植性}
\proname 的网页客户端代码应使用标准的 HTML5 和 Javascript ,而尽量避免使用浏览器特有的 API ;使用响应性网页设计;从而使得在任意分辨率和任意支持 HTML5 的浏览器上都能正常显示。 \proname 的本地客户端应尽可能少的使用最新的和过时的 API ;可以使用跨平台的框架(如 Qt ),并在操作系统变动时能尽可能地减小修改。

\section {可测试性}
\proname 的各个模块间应尽可能地减少外部和内部的依赖。如有依赖,应保证尽可能地使用稳定的 API ,并可以使用 mock object 实现单元测试。

\section {健壮性}
\proname 的客户端和服务器端应检测可能发生的错误并处理。服务器端应保证前端和数据库的隔离,并防止 SQL 注入等攻击方式。

\section {可靠性}
\proname 服务器的数据库应支持高并发的访问,客户端应支持在离线时的正常使用(播放已下载的歌曲),而服务器应有多台,并保证失败时的 failover 。

