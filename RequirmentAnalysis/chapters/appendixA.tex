\chapter{可行性分析结果}

\section{市场分析}

听音乐为一项拥有者广泛的爱好,而通过分享\R{,不仅是自己喜欢的歌曲,还有自己的动态,以及一展自己的歌喉},人们可以找到与自己兴趣相同的人,从而获得集体认同感,而这是人们共有的需求。

现在随着生活和道德水平的提高,人们越来越倾向于使用正版,如购买正版软件、游戏等。而音乐也在此行列之中。拓展音乐版权业务不仅能够满足人们想要支持自己喜欢的歌手的需求,也能给公司带来盈利。

如今歌曲多如牛麻,而其质量则参差不齐。另外,人们也很难从如此大量的歌曲中找到符合自己口味的歌曲。用户创建的歌单通过将相同主体或相近风格的歌曲聚集起来,满足了信息爆炸时代人们的这一需求,同时,创建该歌单的人也达到了分享自己的目的。目前,虽然有许多音乐平台,但这种用户分享歌单的模式仍然在大部分平台上没有得到广泛的应用。

\section{技术可行性分析}

\proname 使用或依赖的软件和技术,如 HTML 、 Javascript 、PostgreSQL , 均为成熟的技术,有稳定的接口和活跃的社区支持。

\proname 的负责人具有使用这些工具的经验。

\section{知识产权分析}

\proname 使用的软件和技术,均基于开源协议发布,或为公开标准,因此不具有知识产权上的侵犯。

库中的曲目\R{,以及用于K歌的背景音乐和MV等}均以合法途径购买\R{并获得使用许可};对用户上传的歌曲,积极进行检查,发现违法的立即下架,则该服务不构成对知识产权的侵犯。\R{通过不允许用户上传非授权歌曲的翻唱,也能防止用户K歌对版权的侵犯。}

